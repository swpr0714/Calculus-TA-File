\documentclass[12px]{article}
\setlength{\parindent}{4em}
\usepackage[margin=2cm]{geometry}
\usepackage{graphicx}
\usepackage{amsmath}
\usepackage{amssymb}
\usepackage{enumerate}
\usepackage{multicol}
\usepackage[font=small,labelfont=bf]{caption}
\usepackage{pifont}
\linespread{1.5}
\begin{document}
\begin{center}
    \Large\textbf{Section 3.5 Derivative and Linear Approximation}
\end{center}
\begin{enumerate}
    \item \textbf{Implicit Differentiation}\\
    In previous chapters, we’re dealing with differentiation of functions in explicit form. However, not every function can be expressed in explicit form, when facing implicit functions, implicit differentiation is applied, examples will be given below.\\
    \textbf{\textit{Example 1}}\\
    Find $y'$ using implicit differentiation, $e^{\frac{x}{y}}=x-y$\\
    \\
    \\
    \\
    \\
    \textbf{\textit{Example 2}}\\
    Find $y''$ using implicit differentiation, $sin(y)+cos(x)=1$\\
    \\
    \\
    \\
    \\
    \textbf{\textit{Exercise 1}}\\
    Differentiate the function with respect to $t$, $x^2cos(y)=sin(y^3+4z)$,\\ assuming that $x=x(t)$, $y=y(t)$, $z=z(t)$.\\
    \\
    \\
    \\
    \\
    \textbf{\textit{Exercise 2}}\\
    If $x^2+xy+y^3=1$, find the value of $y''$ at the point where $x=1$.\\
    \\
    \\
    \\
    \\
    \item \textbf{Derivatives of Logarithmic and Inverse Trigonometric Functions}
    \begin{enumerate}[(1)]
        \item Derivatives of Logarithmic Functions
        $$Let\ f(x)=log_{a}(x),\ f'(x)=\frac{1}{xln(a)}$$
        $$Let\ f(x)=ln(x),\ f'(x)=\frac{1}{x}$$
        \newpage
        \textbf{\textit{Example 1}}\\
        Differentiate $y=\frac{x^{\frac{1}{3}}\sqrt{x^3+1}}{(2x+1)^6}$\\
        \\
        \\
        \\
        \\
        \textbf{\textit{Exercise 1}}\\
        Differentiate $y=(2x-e^{8x})^{sin(2x)}$ (Hint:Remind the domain of the function.)\\
        \\
        \\
        \\
        \\
        \item Derivative of Inverse Trigonometric Functions
        \begin{multicols}{2}
            $\frac{d}{dx}(sin^{-1}x)=\frac{1}{\sqrt{1-x^2}}$\\
            $\frac{d}{dx}(tan^{-1}x)=\frac{1}{1+x^2}$\\
            $\frac{d}{dx}(sec^{-1}x)=\frac{1}{x\sqrt{1-x^2}}$\\
            $\frac{d}{dx}(cos^{-1}x)=\frac{-1}{\sqrt{1-x^2}}$\\
            $\frac{d}{dx}(cot^{-1}x)=\frac{-1}{1+x^2}$\\
            $\frac{d}{dx}(csc^{-1}x)=\frac{-1}{x\sqrt{1-x^2}}$
        \end{multicols}
        \textbf{\textit{Example 1}}\\
        Differentiate $f(x)=4cos^{-1}(x)-10tan^{-1}(x)$\\
        \\
        \\
        \\
        \\
        \\
        \\
        \textbf{\textit{Exercise 1}}\\
        Differentiate $f(x)=tan^{-1}(\frac{x}{a})+ln\sqrt{\frac{x-a}{x+a}}$\\
        \newpage
        \item Number "e" as a limit\\
        \begin{center}
            $e=\lim\limits_{x\to0}(1+x)^{\frac{1}{x}}$\\
            <Proofing method>: Using derivative of $f(x)=ln(x)$\\
            Let $n=\frac{1}{x}$, as $x\to0$, $n\to\infty$, 
            then we can rearrange the equation to:\\
            $e=\lim\limits_{n\to\infty}(1+\frac{1}{n})^{n}$\\
            For higher power of the exponential,\\
            $e^x=\lim\limits_{n\to\infty}(1+\frac{x}{n})^{n}$\\
        \end{center}
    \end{enumerate}
    \item \textbf{Linear Approximations and Differentials}\\
    We had known that if a function is differentiable, then we can use differentiation to find the slope of the tangent line at a certain point of the function. Sometimes we can use tangents to approximate values that are too complex. We call it linear approximation.\\
    \begin{center}
        $f(x)\thickapprox f(a)+f'(a)(x-a) \cdots$\textbf{Figure 1}
    \end{center}
    
    The difference between $dy$ and $\Delta y$:
    \begin{center}
        $
        \left.
        \begin{aligned}
            &dy:\ Approximate\ error,\quad dy = f(a)+f'(a)(\Delta x) \nonumber \\
            &\Delta y:\ Actual\ error,\ \,\quad\qquad \Delta y=f(a+\Delta x)-f(a) \nonumber
        \end{aligned}
        \right\} 
        $ \textbf{Figure 2}
    \end{center}
    \begin{multicols}{2}
        \includegraphics*[height=4cm]{Linearization.png}\captionof{figure}{}
        \includegraphics*[height=4cm]{Lin_Approx.png}\captionof{figure}{}
    \end{multicols}
    \textbf{\textit{Example 1}}\\
    Use a linear approximation to estimate the given number, $ln(sin(29^{\circ}))$\\
    \\
    \\
    \\
    \textbf{\textit{Exercise 1}}\\
    Use a linear approximation to estimate the given number, $\frac{1}{4.002}$
    \newpage
    \textbf{\textit{Example 2}}\\
    If a current $I$ passes through a resistor with resistance $R$, Ohm's Law states that the voltage drop is $V=IR$. If $V$ is constant and $R$ is measured with a certain error, use differentials to show that the relative error of $R$ and $I$ is the same.\\
    \\
    \\
    \\
    \\
    \\
    \\
    \\
    \\
    \\
    \\
    \\
    \textbf{\textit{Exercise 2}}\\
    The circumference of a sphere was measured to be 84 cm with a possible error of 0.5 cm. Use differentials to estimate the maximun error in the calculated volume. What is the relative error.\\
    \\
    \\
    \\
    \\
    \\
    \\
    \\
    

\end{enumerate}
\end{document}