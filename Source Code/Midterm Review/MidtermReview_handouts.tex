\documentclass[12px]{article}
\setlength{\parindent}{4em}
\usepackage[margin=2cm]{geometry}
\usepackage{graphicx}
\usepackage{amsmath}
\usepackage{amssymb}
\usepackage{enumerate}
\usepackage{multicol}
\usepackage[font=small,labelfont=bf]{caption}
\usepackage{pifont}
\usepackage{float}
\usepackage{ulem}
\linespread{1.5}
\begin{document}
\begin{center}
    \Large\textbf{Section 1-4 Midterm Review}
\end{center}
\begin{enumerate}
    \item \textbf{Function}
    \begin{enumerate}[(1)]
        \item Exponential Function ($e\cong 2.7183$,\ $f(x)=e^x,\ f'(0)=e^0=1$)
        \begin{multicols}{4}
            \begin{enumerate}
                \item $b^{x+y}=b^xb^y$
                \item $b^{x-y}={b^x}{b^y}$
                \item $(b^x)^y=b^{xy}$
                \item $(ab)^x=a^xb^x$
            \end{enumerate}
        \end{multicols}
        \item Logarithmic Function ($ln(x) \equiv log_e(x)$.) \textbf{Caution:} Should be written as $ln$ not $In$. \textbf{(Lowercase LN!)}
        \begin{multicols}{4}
            \begin{enumerate}
                \item $\ln(xy)=\ln{x}+\ln{y}$
                \item $\ln(\frac{x}{y})=\ln{x}-\ln{y}$
                \item $\ln(x^r)=r\ln{x}$
                \item $\ln(e^x)=x$
            \end{enumerate}
        \end{multicols}
        \item Inverse Function ($f(x)=y \Leftrightarrow f^{-1}(y)=x$)\\
        Step 1. Write $y=f(x)$.\\
        Step 2. Solve the equation of $x$ in terms of $y$, then we can get $f^{-1}(y)=x$\\
        Step 3. Express the $f^{-1}$ as a function of $x$. (As Required)
        \item Inverse Trigonometric Function
        \begin{align}
            &y = sin^{-1}(x) \Rightarrow &&\text{Domain:}\ -1 \leq x \leq 1,\ &&\text{Range:}\ -\frac{\pi}{2}\leq y \leq\frac{\pi}{2}\nonumber\\
            &y = cos^{-1}(x) \Rightarrow &&\text{Domain:}\ -1 \leq x \leq 1,\ &&\text{Range:}\quad\ \,0\ \leq y \leq\pi\nonumber\\
            &y = tan^{-1}(x) \Rightarrow &&\text{Domain:}\quad x\in \mathbb{R},\ &&\text{Range:}\ -\frac{\pi}{2} \leq y \leq \frac{\pi}{2} \nonumber
        \end{align}
    \end{enumerate}


    \item \textbf{Limit}
    \begin{enumerate}[(1)]
        \item One-side Limit and Existence\\
        \hspace*{2em}If $\lim\limits_{x\to a^+}f(x) = \lim\limits_{x\to a^-}f(x)$, then we can say $\lim\limits_{x\to a}f(x)$ exist.
        \item Asymptotes
        \begin{enumerate}
            \item Vertical Asymptotes: Find the $x$ in the domain of $f(x)$ such that $y$ approach $\pm\infty$.
            \item Horizontal Asymptotes: Check if $f(x)$ approach constant as $x$ approaches $\pm\infty$.
            \item Slant Asymptotes: Factorize the fraction and check if $f(x)$ approach a specific linear function $g(x)$ \\
            as $x$ approach $\pm\infty$.
        \end{enumerate}
        \item Calculation
        \begin{enumerate}
            \item Direct Substitutions: The most intuitive way, just substitute x into the approaching value. 
            \item Fractional Reduction: Factorize or Reduce the fraction then use the substitution. 
            \item Absolute \& Peicewise Function:\\
            E.g.: Guassian Floor or Ceiling Function, Heaviside Unit Step Function.\\
            Discuss different interval using one-side limit.
            \item Squeeze Theorem: \\
            If $g(x)\leq f(x)\leq h(x)$ and $\lim\limits_{x\to a}g(x)=\lim\limits_{x\to a}h(x)=L$, then $\lim\limits_{x\to a}f(x)=L$\\
            Hint: Think of the Squeeze theorem whlie dealing with the limit of trigonometric Function.
            \item L'Hospital Rule: Using to deal with the limit of undeterminate form.\\
            \\
            Suppose 
            $\quad\left\{ 
            \begin{aligned}
                &\lim\limits_{x\to a}\frac{f(x)}{g(x)}=\frac{0}{0}\ or\\ 
                &\lim\limits_{x\to a}\frac{f(x)}{g(x)}=\pm\frac{\infty}{\infty}
            \end{aligned}
            \right.\Rightarrow$
            $\quad
            \begin{aligned}
                &\lim\limits_{x\to a}\frac{f(x)}{g(x)}\ ("\frac{0}{0}")\quad&\overset{L.H.}{=}\quad&\lim\limits_{x\to a}\frac{f'(x)}{g'(x)}\\
                &\lim\limits_{x\to a}\frac{f(x)}{g(x)}\ ("\frac{\infty}{\infty}")\quad&\overset{L.H.}{=}\quad&\lim\limits_{x\to a}\frac{f'(x)}{g'(x)}
            \end{aligned}$\\
            \\
            Type of undeterminate form:
            \begin{enumerate}[(1)]
                \item Fraction: $\frac{0}{0}$, $\frac{\infty}{\infty}$ (Use L'Hospital Rule directly)
                \item Product: $0\cdot\infty$ (Move $0$ or $\infty$ to the denomenator to form $\frac{0}{0}$ or $\frac{\infty}{\infty}$)
                \item Power: $0^0$, $1^\infty$, $\infty^0$ (Try to use Natural Log (ln) to form $\frac{0}{0}$ or $\frac{\infty}{\infty}$)
                \item Subtraction: $\infty-\infty$ (Reduce or Factorize the Square root or Fraction to form $\frac{0}{0}$ or $\frac{\infty}{\infty}$)
            \end{enumerate}
        \end{enumerate}
    \end{enumerate}

    \item \textbf{Continuity}
    \begin{enumerate}[(1)]
        \item Definition
        \begin{multicols}{3}
            \begin{enumerate}
                \item $\lim\limits_{x\to a}f(x)$ exist.
                \item $f(a)$ is defined.
                \item $\lim\limits_{x\to a}f(x)=f(a)$
            \end{enumerate}
        \end{multicols}
        \item Type of Discontinuity
        \begin{multicols}{4}
            \begin{enumerate}
                \item Hole
                \item Infinity
                \item Break
                \item Oscillation
            \end{enumerate}
        \end{multicols}
    \end{enumerate}
    \item \textbf{Derivative}
    \begin{enumerate}[(1)]
        \item Definition\\
        If $\lim\limits_{x\to a^+}\frac{f(x)-f(a)}{x-a}=\lim\limits_{x\to a^-}\frac{f(x)-f(a)}{x-a}\text{, then } f'(x)=\lim\limits_{x\to a}\frac{f(x)-f(a)}{x-a}$
        \item Common Derivative
        \begin{multicols}{3}
            \begin{enumerate}
                \item $\frac{d}{dx}\ (Constant)=0$
                \item $\frac{d}{dx}\ (x^n)=nx^{n-1}$
                \item $\frac{d}{dx}\ (e^x)=e^x$
                \item $\frac{d}{dx}\ (\log_n{x})=\frac{1}{x\ln{n}}$
                \item $\frac{d}{dx}\ (\ln{x})=\frac{1}{x}$
                \item $\frac{d}{dx}\ (sinx)=cosx$
                \item $\frac{d}{dx}\ (cosx)=-sinx$
                \item $\frac{d}{dx}\ (tanx)=sec^2x$
                \item $\frac{d}{dx}\ (secx)=secx\cdot tanx$
            \end{enumerate}
        \end{multicols}
        \item Product Rule and Quotient Rule
        \begin{enumerate}
            \item Product Rule: $\frac{d}{dx}[f(x)\cdot g(x)]=f'(x)g(x)+f(x)g'(x)$
            \item Quotient Rule: $\frac{d}{dx}[\frac{f(x)}{g(x)}]=\frac{f'(x)g(x)-f(x)g'(x)}{[g(x)]^2}$
        \end{enumerate}
        \item Chain Rule: Treat the function as a composite function and differentiate layer by layer.\\
        $\frac{d}{dx}\ f(g(h(x)))=f'(g(h(x)))\cdot g'(h(x))\cdot h'(x)$
        \item Differentiation in graphics
        \begin{enumerate}
            \item First derivatie $f'(x)\Rightarrow$ Slope of tangent $\equiv$ Rate of change. ($=0\Rightarrow$ Critical Point)
            \item First derivatie $f'(x)\Rightarrow$ Concavity of the function. ($=0\Rightarrow$ Inflection Point)
        \end{enumerate}
        \newpage
        \item Implicit Differentiation\\
        Process: Derivative the equation, rearrange the equation into the form $y'(x)=\cdots$
        \item Mean Value Theorem\\
        If $f(x)$ is continuous on $[a,b]$ and differentiable on $(a,b)$, then there exist $c$, s.t. $f'(c)=\frac{f(b)-f(a)}{b-a}$. 
        \item Application
        \begin{enumerate}
            \item Rate of Change: Tangent Slope, $f'(x)=\lim\limits_{\Delta x\to 0}\frac{\Delta y}{\Delta x}$
            \item Linear Approximation: $f(x)\cong f(a)+f'(a)(x-a)$
            \item Related Error: $\frac{\Delta y}{y}\cong \frac{dy}{y}$
            \item Maximum, Minimum and Optimization\\
            \\
            $\left\{
            \begin{aligned}
                    \text{Local Maximum\ (L.M.)}\\
                    \text{Local Minimum\ (L.m.)} 
                \end{aligned}
            \right. \Rightarrow$ Occur while $f'(x)=0$ or D.N.E.\\
            $\left\{
            \begin{aligned}
                    \text{Absolute Maximum\ (A.M.)}\\
                    \text{Absolute Minimum\ (A.m.)} 
                \end{aligned}
            \right. \Rightarrow$ Occur at Local 
            $\left( \begin{aligned}
                \text{Maximum}\\
                \text{Minimum}
            \end{aligned}\right)$
            or End points\\
            \\
            \\
            Procedure of Optimization:
            \begin{enumerate}[{[1]}]
                \item Comprehend the question
                \item List the equation
                \item Find the A.M. or A.m. to solve the question(Check $1^{\text{st}}$ Derivative and the End Points.)
            \end{enumerate}
        \end{enumerate}
    \end{enumerate}
\end{enumerate}
\end{document}