\documentclass[12px]{article}
\setlength{\parindent}{4em}
\usepackage[margin=2cm]{geometry}
\usepackage{graphicx}
\usepackage{amsmath}
\usepackage{amssymb}
\usepackage{enumerate}
\usepackage{multicol}
\usepackage{color}
\usepackage[font=small,labelfont=bf]{caption}
\usepackage{pifont}
\usepackage{float}
\linespread{1.5}
\begin{document}
\begin{center}
    \Large\textbf{Section 7 Techniques of Integration}
\end{center}
\hspace*{2em}In this section, we'll be facing some integrals which its antiderivatives cannot be found intuitively. In this case, the following integral methods will be applied, it will be important for you to identify which method needs to be applied rapidly and precisely by noting their characteristics.
\begin{enumerate}
    \item Variable Substitution\\
    \hspace*{2em}Variable substitution can be realized as a method of divide and conquer, appropriate substitution can greatly simplify the integral.
    \begin{multicols}{2}
    \textit{\textbf{Example 1}}\\
    Evaluate the integral, $\int\frac{1}{e^x+1}dx$\\
    \\
    \\
    \\
    \\
    \\
    \textit{\textbf{Example 2}}\\
    Evaluate the integral, $\int cos^3x\ dx$\\
    \\
    \\
    \\
    \\
    \\
    \textit{\textbf{Example 3}}\\
    Evaluate the integral, $\int\frac{x+sin^{-1}x}{\sqrt{1-x^2}}dx$\\
    \\
    \\
    \\
    \\
    \textit{\textbf{Exercise 1}}\\
    Evaluate the integral, $\int_{1}^{2}(3x-2)^4dx$\\
    \\
    \\
    \\
    \\
    \\
    \textit{\textbf{Exercise 2}}\\
    Evaluate the integral, $\int_{1}^{3}\frac{1}{x\sqrt{4x+1}}dx$\\
    \\
    \\
    \\
    \\
    \\
    \textit{\textbf{Exercise 3}}\\
    Evaluate the integral, $\int10sin(2x)cos(2x)\sqrt{cos^2(2x)-5}\ dx$\\
    \\
    \\
    \\
\end{multicols}
    \item Trigonometric Identities\\
    \hspace*{2em} Except for the basic variable substitution, we'll also use some trigonometric identities to simplify the integrations. The following table enumerates several common trigonometric substitutions.
    \begin{center}
        $\begin{tabular}{ccc}
            \hline
            \hspace{4em}\text{Identities}\hspace{4em} & \hspace{4em}\text{Expression}\hspace{4em} & \hspace{4em}\text{Substitution}\qquad\qquad\qquad\\
            \hline
            $1-sin^2x=cos^2x$ & $\sqrt{\alpha^2-x^2}$ & $x=\alpha sinx$\\
            $1+tan^2x=sec^2x$ & $\sqrt{\alpha^2+x^2}$ & $x=\alpha tanx$\\
            $sec^2x-1=tan^2x$ & $\sqrt{x^2-\alpha^2}$ & $x=\alpha secx$\\
            \hline
            
        \end{tabular}$
    \end{center}

    \newpage
    
    \begin{multicols}{2}
        \textit{\textbf{Example 1}}\\
        Evaluate the integral, $\int\frac{x+3}{\sqrt{4-x^2}}dx$\\
        \\
        \\
        \\
        \\
        \\
        \textit{\textbf{Example 2}}\\
        Evaluate the integral, $\int\sqrt{x-x^2}dx$\\
        \\
        \\
        \\
        \\
        \\
        \textit{\textbf{Exercise 1}}\\
        Evaluate the integral, $\int\frac{2}{x^4\sqrt{x^2-25}}dx$\\
        \\
        \\
        \\
        \\
        \\
        \textit{\textbf{Exercise 2}}\\
        Evaluate the integral, $\int x^2\sqrt{3+2x-x^2}dx$\\
        \\
        \\
        \\
        \\
    \end{multicols}
    \hspace*{2em}In addition to trigonometric substitution, there are still some tricky methods of integration related to trigonometric identities.\\
    \begin{multicols}{2}
        \textit{\textbf{Example 1}}\\
        Evaluate the integral, $\int\frac{1}{cosx+sinx}dx$\\
        \\
        \\
        \\
        \\
        \textit{\textbf{Example 2}}\\
        Evaluate the integral, $\int\frac{cosx-sinx}{2cosx+3sinx}dx$\\
        \\
        \\
        \\
        \\
        \textit{\textbf{Example 3}}\\
        Evaluate the integral, $\int sin^4x\ dx$\\
        \\
        \\
        \\
        \\
        \\
        \textit{\textbf{Exercise 1}}\\
        Evaluate the integral, $\int\frac{1}{cosx+3sinx}dx$\\
        \\
        \\
        \\
        \\
        \textit{\textbf{Exercise 2}}\\
        Evaluate the integral, $\int\frac{2sinx+cosx}{4cosx+sinx}dx$\\
        \\
        \\
        \\
        \\
        \textit{\textbf{Exercise 3}}\\
        Evaluate the integral, $\int cos^2x\ dx$\\
        \\
        \\
        \\
        \\
    \end{multicols}
    \newpage
    \item Integration by Part\\
    \hspace*{2em}Integration by part is an integration technique that treats the integral as the product of $u(x)$ and $v'(x)$, which can greatly simplify the complex integration. The integration by part formula state:
    $$\int_a^b{\color[rgb]{1,0.18,0.18}u(x)}{\color[rgb]{0.278,0.639,1}v'(x)}dx={\color[rgb]{1,0.18,0.18}u(x)}{\color[rgb]{0.278,0.639,1}v(x)}\Big |^b_a-\int_a^b {\color[rgb]{1,0.18,0.18}u'(x)}{\color[rgb]{0.278,0.639,1}v(x)}dx$$
    \begin{multicols}{2}
        \textit{\textbf{Example 1}}\\
        Evaluate the integral, $\int(x^2+2x)cos(x)dx$\\
        \\
        \\
        \\
        \\
        \\
        \\
        \\
        \\
        \textit{\textbf{Example 2}}\\
        Evaluate the integral, $\int_0^{e-3}\ln(x+3)dx$\\
        \\
        \\
        \\
        \\
        \\
        \\
        \\
        \\
        \\
        \textit{\textbf{Example 3}}\\
        Evaluate the integral, $\int sec^3(x)dx$\\
        \\
        \\
        \\
        \\
        \\
        \\
        \\
        \textit{\textbf{Exercise 1}}\\
        Evaluate the integral, $\int(3x+x^2)sin(2x)dx$\\
        \\
        \\
        \\
        \\
        \\
        \\
        \\
        \\
        \textit{\textbf{Exercise 2}}\\
        Evaluate the integral, $\int sin^{-1}(x)dx$\\
        \\
        \\
        \\
        \\
        \\
        \\
        \\
        \\
        \\
        \textit{\textbf{Exercise 3}}\\
        Evaluate the integral, $\int \frac{\ln(x)}{x^2}dx$\\
        \\
        \\
        \\
        \\
        \\
    \end{multicols}
\newpage
    \item Partial Fraction\\
    \hspace*{2em}The partial fraction method is using partial fractal decomposition to simplify the rational function $\frac{R(x)}{Q(x)}$, this type of integration is usually more complex and often accompanied with other techniques.\\
    \hspace*{2em}According to the Fundamation Theorem of Algebra, for all rational function $\frac{R(x)}{Q(x)}$ can be decomposition to the addition of $\frac{a}{(x-b)^n}$ and $\frac{ax+b}{(x^2+cx+d)^m}$. ($n\in\mathbb{R},\ \text{and }c^2-4d<0$), and there are 4 types of partial fractal decomposition.\\
    \begin{multicols}{2}
    No multiplicity:
    \begin{enumerate}[(1)]
        \item $\frac{x^2+2x-1}{x(2x+1)(x+2)}=\frac{A}{x}+\frac{B}{2x-1}+\frac{C}{x+2}$
        \item $\frac{x}{(x-2)(x^2+1)(x^2+4)}=\frac{A}{x-2}+\frac{Bx+C}{x^2+1}+\frac{Dx+E}{x^2+4}$
    \end{enumerate}
    Have at least one multiplicity
    \begin{enumerate}[(1)]\setcounter{enumii}{2}
        \item $\frac{x^3-x+1}{x^2\color[rgb]{1,0.18,0.18}{(x-1)^3}}=\frac{A}{x}+\frac{B}{x^2}+\frac{\color[rgb]{0.278,0.639,1}{C}}{\color[rgb]{1,0.18,0.18}{x-1}}+\frac{\color[rgb]{0.278,0.639,1}{D}}{\color[rgb]{1,0.18,0.18}{(x-1)^2}}+\frac{\color[rgb]{0.278,0.639,1}{E}}{\color[rgb]{1,0.18,0.18}{(x-1)^3}}$
        \item $\frac{-x^3+x^2-x+1}{x\color[rgb]{1,0.18,0.18}{(x^2+1)^2}}=\frac{A}{x}+\frac{\color[rgb]{0.278,0.639,1}{Bx+C}}{\color[rgb]{1,0.18,0.18}{x^2+1}}+\frac{\color[rgb]{0.278,0.639,1}{Dx+E}}{\color[rgb]{1,0.18,0.18}{(x^2+1)^2}}$
    \end{enumerate}
    \end{multicols}
    \begin{multicols}{2}
        \textit{\textbf{Example 1}}\\
        Evaluate the integral, $\int\frac{x^2}{(1+x^2)^2}dx$\\
        \\
        \\
        \\
        \\
        \\
        \\
        \textit{\textbf{Example 2}}\\
        Evaluate the integral, $\frac{dx}{x^2+x\sqrt{x}}$\\
        \\
        \\
        \\
        \\
        \\
        \\
        \textit{\textbf{Exercise 1}}\\
        Evaluate the integral, $\int\frac{1}{x^2+x+1}dx$\\
        \\
        \\
        \\
        \\
        \\
        \\
        \textit{\textbf{Exercise 2}}\\
        Evaluate the integral, $\int \frac{\sqrt{1+\sqrt{x}}}{x}dx$\\
        \\
        \\
        \\
        \\
        \\
    \end{multicols}
    \item Extra Contents:
    \begin{multicols}{2}
        \textit{\textbf{Example 1}}\\
        Evaluate the integral, $\int\frac{1}{1+\varepsilon cos(\theta)}dx$\\
        \\
        \\
        \\
        \\
        \\
        \textit{\textbf{Example 2}}\\
        Evaluate the integral, 
        $\left\{
        \begin{aligned}
            \int e^{ax}cos(bx)dx\\
            \int e^{ax}sin(bx)dx
        \end{aligned}\right.$
        \\
        \\
        \\
        \\
    \end{multicols}
\end{enumerate}
\end{document}